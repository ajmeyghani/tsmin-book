\documentclass[12pt,]{article}
\usepackage{lmodern}
\usepackage{amssymb,amsmath}
\usepackage{ifxetex,ifluatex}
\usepackage{fixltx2e} % provides \textsubscript
\ifnum 0\ifxetex 1\fi\ifluatex 1\fi=0 % if pdftex
  \usepackage[T1]{fontenc}
  \usepackage[utf8]{inputenc}
\else % if luatex or xelatex
  \ifxetex
    \usepackage{mathspec}
    \usepackage{xltxtra,xunicode}
  \else
    \usepackage{fontspec}
  \fi
  \defaultfontfeatures{Mapping=tex-text,Scale=MatchLowercase}
  \newcommand{\euro}{€}
    \setmainfont{Palatino}
    \setsansfont{Century Gothic}
    \setmonofont[Mapping=tex-ansi]{Consolas}
\fi
% use upquote if available, for straight quotes in verbatim environments
\IfFileExists{upquote.sty}{\usepackage{upquote}}{}
% use microtype if available
\IfFileExists{microtype.sty}{%
\usepackage{microtype}
\UseMicrotypeSet[protrusion]{basicmath} % disable protrusion for tt fonts
}{}
\ifxetex
  \usepackage[setpagesize=false, % page size defined by xetex
              unicode=false, % unicode breaks when used with xetex
              xetex]{hyperref}
\else
  \usepackage[unicode=true]{hyperref}
\fi
\hypersetup{breaklinks=true,
            bookmarks=true,
            pdfauthor={Amin Meyghani},
            pdftitle={TypeScript Minified},
            colorlinks=true,
            citecolor=blue,
            urlcolor=blue,
            linkcolor=magenta,
            pdfborder={0 0 0}}
\urlstyle{same}  % don't use monospace font for urls
\usepackage{fancyhdr}
\pagestyle{fancy}
\pagenumbering{arabic}
\lhead{\itshape TypeScript Minified}
\chead{}
\rhead{\itshape{\nouppercase{\leftmark}}}
\lfoot{}
\cfoot{}
\rfoot{\thepage}
\usepackage{color}
\usepackage{fancyvrb}
\newcommand{\VerbBar}{|}
\newcommand{\VERB}{\Verb[commandchars=\\\{\}]}
\DefineVerbatimEnvironment{Highlighting}{Verbatim}{commandchars=\\\{\}}
% Add ',fontsize=\small' for more characters per line
\newenvironment{Shaded}{}{}
\newcommand{\KeywordTok}[1]{\textcolor[rgb]{0.00,0.00,1.00}{{#1}}}
\newcommand{\DataTypeTok}[1]{{#1}}
\newcommand{\DecValTok}[1]{{#1}}
\newcommand{\BaseNTok}[1]{{#1}}
\newcommand{\FloatTok}[1]{{#1}}
\newcommand{\ConstantTok}[1]{{#1}}
\newcommand{\CharTok}[1]{\textcolor[rgb]{0.00,0.50,0.50}{{#1}}}
\newcommand{\SpecialCharTok}[1]{\textcolor[rgb]{0.00,0.50,0.50}{{#1}}}
\newcommand{\StringTok}[1]{\textcolor[rgb]{0.00,0.50,0.50}{{#1}}}
\newcommand{\VerbatimStringTok}[1]{\textcolor[rgb]{0.00,0.50,0.50}{{#1}}}
\newcommand{\SpecialStringTok}[1]{\textcolor[rgb]{0.00,0.50,0.50}{{#1}}}
\newcommand{\ImportTok}[1]{{#1}}
\newcommand{\CommentTok}[1]{\textcolor[rgb]{0.00,0.50,0.00}{{#1}}}
\newcommand{\DocumentationTok}[1]{\textcolor[rgb]{0.00,0.50,0.00}{{#1}}}
\newcommand{\AnnotationTok}[1]{\textcolor[rgb]{0.00,0.50,0.00}{{#1}}}
\newcommand{\CommentVarTok}[1]{\textcolor[rgb]{0.00,0.50,0.00}{{#1}}}
\newcommand{\OtherTok}[1]{\textcolor[rgb]{1.00,0.25,0.00}{{#1}}}
\newcommand{\FunctionTok}[1]{{#1}}
\newcommand{\VariableTok}[1]{{#1}}
\newcommand{\ControlFlowTok}[1]{\textcolor[rgb]{0.00,0.00,1.00}{{#1}}}
\newcommand{\OperatorTok}[1]{{#1}}
\newcommand{\BuiltInTok}[1]{{#1}}
\newcommand{\ExtensionTok}[1]{{#1}}
\newcommand{\PreprocessorTok}[1]{\textcolor[rgb]{1.00,0.25,0.00}{{#1}}}
\newcommand{\AttributeTok}[1]{{#1}}
\newcommand{\RegionMarkerTok}[1]{{#1}}
\newcommand{\InformationTok}[1]{\textcolor[rgb]{0.00,0.50,0.00}{{#1}}}
\newcommand{\WarningTok}[1]{\textcolor[rgb]{0.00,0.50,0.00}{\textbf{{#1}}}}
\newcommand{\AlertTok}[1]{\textcolor[rgb]{1.00,0.00,0.00}{{#1}}}
\newcommand{\ErrorTok}[1]{\textcolor[rgb]{1.00,0.00,0.00}{\textbf{{#1}}}}
\newcommand{\NormalTok}[1]{{#1}}
\setlength{\parindent}{0pt}
\setlength{\parskip}{6pt plus 2pt minus 1pt}
\setlength{\emergencystretch}{3em}  % prevent overfull lines
\providecommand{\tightlist}{%
  \setlength{\itemsep}{0pt}\setlength{\parskip}{0pt}}
\setcounter{secnumdepth}{5}

\title{TypeScript Minified}
\author{Amin Meyghani}
\date{}

% Redefines (sub)paragraphs to behave more like sections
\ifx\paragraph\undefined\else
\let\oldparagraph\paragraph
\renewcommand{\paragraph}[1]{\oldparagraph{#1}\mbox{}}
\fi
\ifx\subparagraph\undefined\else
\let\oldsubparagraph\subparagraph
\renewcommand{\subparagraph}[1]{\oldsubparagraph{#1}\mbox{}}
\fi

\begin{document}
\maketitle

{
\hypersetup{linkcolor=black}
\setcounter{tocdepth}{5}
\tableofcontents
}
\section{Introduction}\label{introduction}

This is a book from the \emph{Minifed} series on TypeScript. It goes
through the essentials very fast so that you can get up to speed with
TypeScript. The theme of this book is TypeScript and Angular2.

\section{Object Orientation}\label{object-orientation}

Interfaces and classes are heavily used in Object Oriented Programming.
In this chapter we will focus on these topics.

\section{Interface}\label{interface}

\begin{itemize}
\tightlist
\item
  An Interface is defined using the \texttt{interface} keyword
\item
  Interfaces are used only during compilation time to check types
\item
  By convention, interface definitions start with an \texttt{I}, e.g. :
  \texttt{IPoint}
\item
  Interfaces are used in classical object oriented programming as a
  design tool
\item
  Interfaces don't contain implementations
\item
  They provide definitions only
\item
  When an object implements an interface, it must adhere to the contract
  defined by the interface
\item
  An interface defines what properties and methods an object must
  implement
\item
  If an object implements an interface, it must adhere to the contract.
  If it doesn't the compiler will let us know.
\item
  Interfaces also define custom types
\end{itemize}

\subsection{Basic Interface}\label{basic-interface}

Below is an example of an Interface that defines two properties and
three methods that implementers should provide implementations for:

\begin{Shaded}
\begin{Highlighting}[numbers=left,,]
\KeywordTok{interface} \NormalTok{IMyInterface \{}
  \CommentTok{// some properties}
  \NormalTok{id: number;}
  \NormalTok{name: string;}

  \CommentTok{// some methods}
  \FunctionTok{method}\NormalTok{(): }\DataTypeTok{void}\NormalTok{;}
  \FunctionTok{methodWithReturnVal}\NormalTok{():number;}
  \FunctionTok{sum}\NormalTok{(nums: number[]):number;}
\NormalTok{\}}
\end{Highlighting}
\end{Shaded}

Using the interface above we can create an object that adheres to the
interface:

\begin{Shaded}
\begin{Highlighting}[numbers=left,,]
\NormalTok{let myObj: IMyInterface = \{}
  \NormalTok{id: }\DecValTok{2}\NormalTok{,}
  \NormalTok{name: 'some name',}

  \FunctionTok{method}\NormalTok{() \{ console.}\FunctionTok{log}\NormalTok{('hello'); \},}
  \FunctionTok{methodWithReturnVal} \NormalTok{() \{ }\KeywordTok{return} \DecValTok{2}\NormalTok{; \},}
  \FunctionTok{sum}\NormalTok{(numbers) \{}
    \KeywordTok{return} \NormalTok{numbers.}\FunctionTok{reduce}\NormalTok{( (a,b) => \{ }\KeywordTok{return} \NormalTok{a + b \} );}
  \NormalTok{\}}
\NormalTok{\};}
\end{Highlighting}
\end{Shaded}

Notice that we had to provide values to \textbf{all} the properties
defined by the Interface, and the implementations for \textbf{all} the
methods defined by the Interface.

And then of course you can use your object methods to perform
operations:

\begin{Shaded}
\begin{Highlighting}[numbers=left,,]
\NormalTok{let sum = myObj.}\FunctionTok{sum}\NormalTok{([}\DecValTok{1}\NormalTok{,}\DecValTok{2}\NormalTok{,}\DecValTok{3}\NormalTok{,}\DecValTok{4}\NormalTok{,}\DecValTok{5}\NormalTok{]); }\CommentTok{// -> 15}
\end{Highlighting}
\end{Shaded}

\subsection{Some Angular Interfaces}\label{some-angular-interfaces}

Angular uses interfaces all over the place. The interfaces that are used
very often are the \emph{LifeCycle Hooks}.

\subsubsection{LifeCycle Interfaces}\label{lifecycle-interfaces}

\begin{Shaded}
\begin{Highlighting}[numbers=left,,]
\NormalTok{export }\KeywordTok{interface} \NormalTok{OnChanges \{}
  \FunctionTok{ngOnChanges}\NormalTok{(changes: \{}
    \NormalTok{[key: string]: SimpleChange}
  \NormalTok{\});}
\NormalTok{\}}

\NormalTok{export }\KeywordTok{interface} \NormalTok{OnInit \{}
  \FunctionTok{ngOnInit}\NormalTok{();}
\NormalTok{\}}

\NormalTok{export }\KeywordTok{interface} \NormalTok{OnDestroy \{}
  \FunctionTok{ngOnDestroy}\NormalTok{();}
\NormalTok{\}}
\end{Highlighting}
\end{Shaded}

\section{Classes}\label{classes}

\begin{itemize}
\tightlist
\item
  Classes are heavily used in classical object oriented programming
\item
  It defines what an object is and what it can do
\item
  A class is defined using the \texttt{class} keyword followed by a name
\item
  By convention, the name of the class start with an uppercase letter
\item
  A class can be used to create multiple objects (instances) of the same
  class
\item
  An object is created from a class using the \texttt{new} keyword
\item
  A class can have a \texttt{constructor} which is called when an object
  is made from the class
\item
  Properties of a class are called instance variables and its functions
  are called the class methods
\item
  Access modifiers can be used to make them public or private
\item
  The instance variables are attached to the instance itself but not the
  prototype
\item
  Methods however are attached to the prototype object as opposed to the
  instance itself
\item
  Classes can inherit functionality from other classes, but you should
  \href{https://medium.com/javascript-scene/the-two-pillars-of-javascript-ee6f3281e7f3\#.oc5pdevwh}{favor
  composition over inheritance} or make sure you know
  \href{https://medium.com/@dtinth/es6-class-classical-inheritance-20f4726f4c4\#.xdif2m42e}{when
  to use it}
\item
  Classes can implement interfaces
\end{itemize}

Let's make a class definition for a car and incrementally add more
things to it.

\subsection{Distance Instance
Variable}\label{distance-instance-variable}

The \texttt{Car} class definition can be very simple and can define only
a single instance variable that all cars can have:

\begin{Shaded}
\begin{Highlighting}[numbers=left,,]
\KeywordTok{class} \NormalTok{Car \{}
  \NormalTok{distance: number;}
\NormalTok{\}}
\end{Highlighting}
\end{Shaded}

\begin{itemize}
\tightlist
\item
  \texttt{Car} is the name of the class, which also defines the custom
  type \texttt{Car}
\item
  \texttt{distance} is a property that tracks the distance that car has
  traveled
\item
  Distance is of type \texttt{number} and only accepts \texttt{number}
  type.
\end{itemize}

Now that we have the definition for a car, we can create a car from the
definition:

\begin{Shaded}
\begin{Highlighting}[numbers=left,,]
\NormalTok{let myCar:Car = }\KeywordTok{new} \FunctionTok{Car}\NormalTok{();}
\NormalTok{myCar.}\FunctionTok{distance} \NormalTok{= }\DecValTok{0}\NormalTok{;}
\end{Highlighting}
\end{Shaded}

\begin{itemize}
\tightlist
\item
  \texttt{myCar:Car} means that \texttt{myCar} is of type \texttt{Car}
\item
  \texttt{new\ Car()} creates an instance from the \texttt{Car}
  definition.
\item
  \texttt{myCar.distance\ =\ 0} sets the initial value of the
  \texttt{distance} to 0 for the newly created \texttt{car}
\end{itemize}

\subsection{Adding a Method}\label{adding-a-method}

So far our car doesn't have any definitions for any actions. Let's
define a \texttt{move} method that all the cars can have:

\begin{Shaded}
\begin{Highlighting}[numbers=left,,]
\KeywordTok{class} \NormalTok{Car \{}
  \NormalTok{distance: number;}
  \FunctionTok{move}\NormalTok{():}\DataTypeTok{void} \NormalTok{\{}
    \KeywordTok{this}\NormalTok{.}\FunctionTok{distance} \NormalTok{+= }\DecValTok{1}\NormalTok{;}
  \NormalTok{\};}
\NormalTok{\}}
\end{Highlighting}
\end{Shaded}

\begin{itemize}
\tightlist
\item
  \texttt{move():void} means that \texttt{move} is a method that does
  not return any value, hence \texttt{void}.
\item
  The body of the method is defined in \texttt{\{\ \}}
\item
  \texttt{this} refers to the instance, therefore \texttt{this.distance}
  points to the \texttt{distance} property defined on the car instance.
\item
  Now you can call the \texttt{move} method on the car instance to
  increment the \texttt{distance} value by 1:
\end{itemize}

\begin{Shaded}
\begin{Highlighting}[numbers=left,,]
\NormalTok{myCar.}\FunctionTok{move}\NormalTok{();}
\NormalTok{console.}\FunctionTok{log}\NormalTok{(myCar.}\FunctionTok{distance}\NormalTok{) }\CommentTok{// -> 1}
\end{Highlighting}
\end{Shaded}

\subsection{Adding a constructor}\label{adding-a-constructor}

A \texttt{constructor} is a special method that gets called when an
instance is created from a class. Let's add a constructor to the
\texttt{Car} class that initializes the \texttt{distance} value to 0.
This means that all the cars that are crated from this class, will have
their \texttt{distance} set to 0 automatically:

\begin{Shaded}
\begin{Highlighting}[numbers=left,,]
\KeywordTok{class} \NormalTok{Car \{}
  \NormalTok{distance: number;}
  \FunctionTok{constructor} \NormalTok{() \{}
    \KeywordTok{this}\NormalTok{.}\FunctionTok{distance} \NormalTok{= }\DecValTok{0}\NormalTok{;}
  \NormalTok{\};}
  \FunctionTok{move}\NormalTok{():}\DataTypeTok{void} \NormalTok{\{}
    \KeywordTok{this}\NormalTok{.}\FunctionTok{distance} \NormalTok{+= }\DecValTok{1}\NormalTok{;}
  \NormalTok{\};}
\NormalTok{\}}
\end{Highlighting}
\end{Shaded}

\begin{itemize}
\tightlist
\item
  \texttt{constructor()} is called automatically when a new car is
  created
\item
  The body of the constructor is defined in the \texttt{\{\ \}}
\end{itemize}

So now when we create a car, the \texttt{distance} property is
automatically set to 0.

\subsection{Using Access Modifiers}\label{using-access-modifiers}

If you wanted to tell the compiler that the \texttt{distance} variable
is private and can only be used by the object itself, you can use the
\texttt{private} modifier before the name of the property:

\begin{Shaded}
\begin{Highlighting}[numbers=left,,]
\KeywordTok{class} \NormalTok{Car \{}
  \KeywordTok{private} \NormalTok{distance: number;}
  \FunctionTok{constructor} \NormalTok{() \{}
    \NormalTok{...}
  \NormalTok{\};}
  \NormalTok{...}
\NormalTok{\}}
\end{Highlighting}
\end{Shaded}

Access modifiers can be used in different places. Check out the access
modifiers chapter for more details.

\subsection{Implementing an Interface}\label{implementing-an-interface}

Classes can implement one or multiple interfaces. We can make the
\texttt{Car} class implement two interfaces:

\textbf{interfaces}

\begin{Shaded}
\begin{Highlighting}[numbers=left,,]
\KeywordTok{interface} \NormalTok{ICarProps \{}
  \NormalTok{distance: number;}
\NormalTok{\}}
\KeywordTok{interface} \NormalTok{ICarMethods \{}
  \FunctionTok{move}\NormalTok{():}\DataTypeTok{void}\NormalTok{;}
\NormalTok{\}}
\end{Highlighting}
\end{Shaded}

Making the \texttt{Car} class implement the interfaces:

\begin{Shaded}
\begin{Highlighting}[numbers=left,,]
\KeywordTok{class} \NormalTok{Car }\KeywordTok{implements} \NormalTok{ICarProps, ICarMethods \{}
  \NormalTok{distance: number;}
  \FunctionTok{constructor} \NormalTok{() \{}
    \KeywordTok{this}\NormalTok{.}\FunctionTok{distance} \NormalTok{= }\DecValTok{5}\NormalTok{;}
  \NormalTok{\};}
  \FunctionTok{move}\NormalTok{():}\DataTypeTok{void} \NormalTok{\{}
    \KeywordTok{this}\NormalTok{.}\FunctionTok{distance} \NormalTok{+= }\DecValTok{1}\NormalTok{;}
  \NormalTok{\};}
\NormalTok{\}}
\end{Highlighting}
\end{Shaded}

The above example is silly, but it shows the point that a class can
implement one or more interfaces. Now if the class does not provide
implementations for any of the interfaces, the compiler will complain.
For example, if we leave out the \texttt{distance} instance variable,
the compiler will print out the following error:

\begin{quote}
error TS2420: Class `Car' incorrectly implements interface `ICarProps'.
Property `distance' is missing in type `Car'.
\end{quote}

\section{Angular and TypeScript}\label{angular-and-typescript}

It is much easier to write Angular with TypeScript.

\subsection{Sample}\label{sample}

A sample ts class with decoration:

\begin{Shaded}
\begin{Highlighting}[numbers=left,,]
\KeywordTok{import \{bootstrap\} from 'angular';}
\FunctionTok{@component}\NormalTok{(\{}
  \NormalTok{selector: 'app'}
\NormalTok{\});}
\KeywordTok{class} \NormalTok{App \{\}}
\end{Highlighting}
\end{Shaded}

The corresponding html

\begin{Shaded}
\begin{Highlighting}[numbers=left,,]
\DataTypeTok{<!DOCTYPE }\NormalTok{html}\DataTypeTok{>}
\KeywordTok{<html>}
\KeywordTok{<head>}
  \KeywordTok{<title>}\NormalTok{example}\KeywordTok{</title>}
\KeywordTok{</head>}
\KeywordTok{<body>}
  \KeywordTok{<app} \ErrorTok{[prop]}\OtherTok{=}\StringTok{'data'}\KeywordTok{></app>}
\KeywordTok{</body>}
\KeywordTok{</html>}
\end{Highlighting}
\end{Shaded}

The corresponding css:

\begin{Shaded}
\begin{Highlighting}[numbers=left,,]
\FloatTok{.app} \KeywordTok{\{}
  \KeywordTok{display:} \DataTypeTok{block}\KeywordTok{;}
\KeywordTok{\}}
\end{Highlighting}
\end{Shaded}

\section{Angular Components}\label{angular-components}

\begin{itemize}
\tightlist
\item
  Almost everything is a component
\end{itemize}

\begin{Shaded}
\begin{Highlighting}[numbers=left,,]
\FunctionTok{@component}\NormalTok{(\{\});}
\KeywordTok{class} \NormalTok{MyComponent \{\}}
\end{Highlighting}
\end{Shaded}

\end{document}
